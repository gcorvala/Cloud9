\chapter{Pré-requis}
%\section{Acquisition \label{acquisition}}
%La phase d'acquisition consiste à l'enregistrement de donnée à l'aide de senseurs. Le résultat de cette acquisition est une carte de distance par rapport à la position du matériel d'acquisition, qui, quant à elle, est présumée connue.
%\subsection{Triangulation}
%(\textbf{Reverse engeneering of geometric models—an introduction})\\

%La méthode par triangulation consiste à déduire la position d'un élément en utilisant la position et l'angle entre plusieurs dispositifs lumineux et un appareillage photo-sensible.

%(\underline{Illustration})

%\subsection{Stéréoscopie}
%La stéréoscopie est basé sur deux appareils d'acquisition, comme deux appareils photos. De cette façon, il est possible de déduire la position des éléments en comparant deux prises de vue différentes.

%\subsection{Holographie conoscopique}
%(\textbf{Holographi conoscopique. Reconstruction numérique})

%\subsection{Caméra temps de vol}
%Les caméras à temps de vol projette de la lumière vers l'objet d'intéret, et calcule la distance entre celui-ci et l'appareil d'acquisition à partir du temps que prend la lumière pour faire le trajet objet-caméra. Ce dispositif à l'avantage d'être extrêmement précis car différentes longueurs d'ondes peuvent être considérées.

%\subsection{Alternatives}
%Toutes les méthodes citées précédement sont basées sur des propriétés optiques. Il existe d'autres méthodes, dites \emph{tactiles} qui permettent également l'acquisition de carte de distance. (\textbf{Reverse engeneering of geometric models—an introduction})

%\section{Recalage d'image \label{registration}\index{Patrimoine culturel}~\cite{greenwade93}}

\chapter{L'échantillonage}

\paragraph
Après l'acquisition \ref{acquisition}, suivi d'une éventuelle phase de recalage \ref{registration}, la quantité d'information recueillie peut rapidement croître.

En effet les nuages de points \index{nuage de points} décrivent souvent des surfaces complexe, et de ce fait, ceux-ci contiennent plusieurs millions, voir des milliards de points~\cite{Levoy}.

En fonction du but recherché, cette masse de donnée peut handicaper le traitement futur, en effet, beaucoup d'algorithmes de traitement d'image n'ont pas une complexité en $O(n)$ et de ce fait, peuvent être trop lent.

Réduire la complexité de ces nuages de points \index{nuage de points} est une étapes importante afin de préparer ces informations a être traité plus ou moins rapidement, en fonction de l'objectif recherché. Le sous-échantillonnage consiste à déterminer un ensemble de points réduit approchant au mieux le nuage de points \index{nuage de points} original.

Voici une définition possible du sous-échantillonnage \index{échantillonnage} \cite{Pauly2003}:

\begin{definition}[Sous-échantillonnage]
  Soit $S$ une surface définie par un nuage de points $P$.
  Soit $n$ le nombre de point dans le nuage sous-échantillonné tel que $n<|P|$, trouver un nuage de points $P'$ tel que $|P'|=n$ de tel sorte que la distance $\epsilon = d(S,S')$ de la surface correspondante $S'$ à la surface originale $S$ est minimale.
\end{definition}

Cette définition peut-être inversée, de façon à chercher un sous ensemble minimal de point tel que la distance entre les deux surfaces $S$ et $S'$ soit minimale.

L'approximation optimale d'une surface est un problème \emph{NP-Complet} \cite{Agarwal1994}, et de ce fait, la plupart des recherches dans le domaine sont orientés vers des heuristiques.

\paragraph
Afin de faciliter la présentation de différentes méthodes liées au sous-échantillonnage de nuage de points, on distinguera différentes approches ~\cite{Pauly2002}:

\begin{itemize}
  \item Par grappe;
  \item Par simplification itérative;
  \item Par simulation de particule.
\end{itemize}

\subsection{Méthode par grappe}
\begin{definition}[Sous-échantillonnage par grappe]
  Consiste à diviser l'ensemble des points en un ensemble de sous-ensemble. Ensuite chacun de ceux-ci est remplacé par un point représentatif.
\end{definition}

Une solution simple permettant d'obtenir des \emph{grappes} consiste à diviser la boite englobante du nuage de point en cellule à l'aide d'un maillage régulier. Chacune d'elle contiendra un ensemble de points, qu'il faudra remplacer par un nouveau point représentatif.
Le problème de cette méthode est lié au maillage régulier. En effet, celui-ci ne peut pas s'adapter au irrégularité dans la répartition des points. De plus, si la taille du maillage est trop grande, il est possible que la surface sous-échantillonnée connecte des éléments qui devraient être disjoints.

Afin d'éviter ce genre de problème, il est nécessaire de grouper les points en respectant certains critères liés aux propriétés de la surface. Il s'agit d'une étape de \emph{segmentation} \index{segmentation}. \'{A} cette fin, on peut distinguer deux approches:

\begin{itemize}
  \item Incrémentale;
  \item Hiérarchique.
\end{itemize}

\subsubsection{Méthode par grappe incrémentale}

Ces méthodes se basent sur le principe de \emph{croissance de région} \index{Croissance de région}.

\subsubsection{Méthode par grappe hiérarchique}

\subsection{Méthode iterative}
\begin{definition}[Sous-échantillonnage par simplification itérative]
  Méthode d'échantillonnage consistant à fusionner successivement des pairs de points dans un nuage de points en fonction d'une mesure d'erreur quadratique.
\end{definition}

\subsection{Methode par simulation}
\begin{definition}[Sous-échantillonnage par simulation de particule]
  Méthode d'échantillonnage consistant a calculer les nouvelles positions des points de l'ensemble $P'$ sur la surface $S$ en respectant des forces inter-particules.
\end{definition}

\section{Reconnaissance de forme}
\section{Reconstruction 3D}
\section{Rendu}