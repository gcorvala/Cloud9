\chapter{Acquisition \label{acquisition}}
La phase d'acquisition consiste à l'enregistrement de donnée à l'aide de senseurs. Le résultat de cette acquisition est une carte de distance par rapport à la position du matériel d'acquisition, qui, quant à elle, est présumée connue.
\subsection{Triangulation}
%(\textbf{Reverse engeneering of geometric models—an introduction})\\

%La méthode par triangulation consiste à déduire la position d'un élément en utilisant la position et l'angle entre plusieurs dispositifs lumineux et un appareillage photo-sensible.

%(\underline{Illustration})

\subsection{Stéréoscopie}
%La stéréoscopie est basé sur deux appareils d'acquisition, comme deux appareils photos. De cette façon, il est possible de déduire la position des éléments en comparant deux prises de vue différentes.

\subsection{Holographie conoscopique}
%(\textbf{Holographi conoscopique. Reconstruction numérique})

\subsection{Caméra temps de vol}
%Les caméras à temps de vol projette de la lumière vers l'objet d'intéret, et calcule la distance entre celui-ci et l'appareil d'acquisition à partir du temps que prend la lumière pour faire le trajet objet-caméra. Ce dispositif à l'avantage d'être extrêmement précis car différentes longueurs d'ondes peuvent être considérées.

\subsection{Alternatives}
%Toutes les méthodes citées précédement sont basées sur des propriétés optiques. Il existe d'autres méthodes, dites \emph{tactiles} qui permettent également l'acquisition de carte de distance. (\textbf{Reverse engeneering of geometric models—an introduction})