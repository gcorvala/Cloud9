\chapter{Introduction}
\setcounter{page}{1}
\begin{quotation}
\noindent ``\emph{certains aiment insérer une ou des citations particulières, en rapport avec le chapitre.}''
\begin{flushright}\textbf{auteur, date}\end{flushright}
\end{quotation}

\vspace*{0.5cm}

\section{Contexte et objectifs du mémoire}
La numérisation 3D de sites ou des bâtiments est de plus en plus utilisée dans les domaines de la conservation du patrimoine ou du génie civil. Pour ce dernier, il est souvent nécessaire de pouvoir récupérer des plans fiables et fidèles d'une infrastructure.\\

Le but de ce mémoire est de fournir un outil informatique automatique ou semi-automatique permettant d'obtenir des plans d'architectes cotés à partir de données de numérisation 3D disponibles sous la forme de nuages de points.\\

Après avoir identifié les techniques issues de l'état de l'art dans le domaine de l'analyse d'image et la reconnaissance de formes, une mise en oeuvre sur des données concrètes acquise dans un bunker sera proposée.\\

Une validation de l'application et des méthodes pourront consister en la numérisation d'une partie d'un bâtiment (par exemple le bâtiment Solvay) et en la comparaison des résultats obtenus avec les plans d'architectes disponibles.

\section{Patrimoine culturel}

\section{Structure du mémoire}
Dans le chapitre \ref{stateOfTheArt}, j'introduirai la notion de nuage de point, ainsi qu'un état de l'art relatif à l'acquisition et aux différents traitement de nuages de point.
Blabla

\section{Contributions du mémoire}
\noindent
La liste suivante présente nos principales contributions :
\vspace{1cm}
\begin{enumerate}
    \item j'espère qu'il y en aura beaucoup~(Chapitres~\ref{chap2} et untels).
    \item et encore~(Section~\ref{sec-untel}).
    \item et encore~(Sections untel, untel et untel).
\end{enumerate}

\clearpage
\section{Notations}